\documentclass[format=acmlarge, review=true, authordraft=true]{acmart}

%% scheme-list :
\usepackage{listings}
\usepackage{color}
\usepackage{textcomp}

\lstset{
  language=Scheme,
  basicstyle=\ttfamily,
  morekeywords={run,conde,run*,defrel,==,fresh},
  alsodigit={!\$\%&*+-./:<=>?@^_~},
  morecomment=[l]{;,\#{}lang},
  mathescape=true
}
%% scheme-list .

% metadata

\title{BFS search in miniKanren}
\author{Kuang-Chen Lu}
\affiliation{Indiana University}
\author{Weixi Ma}
\affiliation{Indiana University}
\author{Daniel P. Friedman}
\affiliation{Indiana University}



%%% NOTE %%%
%
% - plan paper writing
%   
%   * Abstract
%   * send to Dan
%   * Core Content
%     * how stream is implemented in TRS2's miniKanren
%     * how to do BFS search
%
% - when is the deadline
%
%   last year submission deadline July 23, 2018
%
%%%%%%%%%%%%



% body
\begin{document}

\begin{abstract}

The syntax of a programming language should reflect its semantics.
When using a disjunction operator in relational programming, a programmer would
expect all clauses of this disjunction to share the same chance of being explored,
as these clauses are written in parallel. The existing disjunctive operators in miniKanren,
however, prioritize their clauses by the order of which these clauses are written down.
We have devised a new search strategy that searches evenly in all clauses.
Based on our statistics, miniKanren slows down by a constant factor after applying our search strategy.
(tested with very-recursiveo, need more tests)


\end{abstract}

\maketitle

\section{introduction}

OUTLINE:

( About miniKanren )

( Why the left clauses are explored more frequently? )

%TODO define iDFS = interleaving DFS in Introduction

( How to solve the problem? )

( Summary of later sections )

\section{cost of answers}

%TODO the meaning of resolution step in Silvija Seres's paper

The \emph{cost} of an answer is the number of relation applications needed to find the answer. We use the miniKanren relation \texttt{repeato} in Fig.~\ref{def-repeato} to demonstrate the cost of answers. We borrow this idea from Silvija Seres's work [*]. \texttt{repeato} relates \texttt{x} with a list whose elements are all \texttt{x}s. The run expression uses \texttt{repeato} to generate 4 lists whose elements are all \texttt{*}s. The order of the answers reflects the order miniKanren discovers them. The leftmost answer is discovered first. The order here is not suprising: to generate the answer \texttt{'()}, miniKanren needs to apply \texttt{repeato} only once. And it needs more applications of \texttt{repeato} to find the later answers that are more complicated. In this example, the cost of each answer is the same as one more than the number of \texttt{*}s. No answer of zero cost exists.

\begin{figure}
  \lstinputlisting{Figures/defrel-repeato.rkt}
  \lstinputlisting{Figures/run-repeato-1.rkt}
  \caption{\texttt{repeato}}
  \label{def-repeato}
\end{figure}

%TODO relation *application*

For the above \texttt{run}, both search strategies produces answers in increasing order of costs, i.e. both of them are \emph{cost-respecting}. In more complicated cases, however, interleaving DFS does not always produces answers in cost-repecting order. For instance, with iDFS the \texttt{run} in Fig.~\ref{conde-repeato-iDFS} produces answers in a seemingly random order. In contrast, the same run with BFS produces answers in an expected order (Fig.~\ref{conde-repeato-BFS}).

\begin{figure}
	\lstinputlisting{Figures/run-conde-repeato-iDFS.rkt}
	\caption{run a program with interleaving depth-first search}
	\label{conde-repeato-iDFS}
\end{figure}

\begin{figure}
	\lstinputlisting{Figures/run-conde-repeato-BFS.rkt}
	\caption{run a program with breadth-first search}
	\label{conde-repeato-BFS}
\end{figure}

Although all answers by iDFS is not in cost-respecting order, the answers from each case are in cost-respecting order. The problem is that iDFS strategy prioritizes the first \texttt{conde} case considerablely. In general, when every conde case are equally productive, the iDFS strategy takes $1/2^{i}$ answers from the $i$-th case, except the last case, which share the same portion as the second last.




\section{change search strategy}

Now we change the search strategy and optimize the system. The whole process is completed in three steps, corresponding to 4 versions of miniKanren. The initial version, \texttt{mk-0} is exactly the miniKanren in \emph{The Reasoned Schemer, 2nd Edition}.

\subsection{from \texttt{mk-0} to \texttt{mk-1}}

In mk-0, search spaces are represented by streams of answers. Streams can be finite or infinite. Finite streams are just lists. And infinite streams are improper lists, whose last \texttt{cdr} is a thunk returning another stream. We call the \texttt{car}s the \emph{mature} part, and the last \texttt{cdr} the \emph{immature} part. 

Streams are cost respective when they are initially constructed by \texttt{==}. However, \texttt{append-inf}, a composer of stream, breaks cost respectiveness. By changing its definition, we can make the whole search strategy cost respective.

(explain mk-0's append-inf)

\begin{figure}
	 \lstinputlisting{Figures/append-inf-0.rkt}	
	 \caption{\texttt{append-inf} in \texttt{mk-0}}
	 \label{append-inf-0}
\end{figure}

\begin{figure}
	 \lstinputlisting{Figures/append-inf-1.rkt}	
	 \caption{\texttt{append-inf} in \texttt{mk-1}}
	 \label{append-inf-1}
\end{figure}



\subsection{stepstone of optimization}

Make irrelevant parts in mK representation-independent w.r.t. search space, and combine mature part and immature part with cons.

\subsection{optimization}

The goal is to express BFS explicitly with queue, so that we don’t have to generate all answers of the same cost at once.

Interesting changes: (1) put thunks in a list; (2) change force-inf (introduced in 4.B) so that it can make progress in all thunks (3) use a queue to manage thunks in take-inf.

\section{conclusion}

\section*{acknowledgments}

\section*{references}

\end{document}

