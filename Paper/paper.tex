\documentclass[format=acmlarge, review=true, authordraft=true]{acmart}

\usepackage{color}
\usepackage{textcomp}
\usepackage{hyperref}

\usepackage{trackchanges}

\addeditor{MVC}
\addeditor{LKC}
\addeditor{DPF}

\usepackage{listings}
\lstset{
  language=Scheme,
  basicstyle=\ttfamily,
  morekeywords={run,conde,run*defrel,==,fresh,disj,conj,disj+},
  alsodigit={!\$\%&*+-./:<=>?@^_~},
  morecomment=[l]{;,\#{}lang},
  mathescape=true,
  literate={-inf}{{$^\infty$}}{1} {/fair}{{$_{fair}$}}{3}
  		   {→}{{$\rightarrow$}}{1}
  		   {×}{{$\times$}}{1}
  		   {==}{{$\equiv$}}{1}
  		   {conde}{{\textcolor[rgb]{0,.3,.7}{\textbf{cond$^e$}}}}{4}
           {appendo}{{append$^o$}}{6}
  		   {repeato}{{repeat$^o$}}{6}
  		   {disj2}{{disj$_2$}}{4}
  		   {conj2}{{conj$_2$}}{4}
  		   {g0}{{g$_0$}}{2}
  		   {g1}{{g$_1$}}{2}
  		   {g2}{{g$_2$}}{2}
  		   {g3}{{g$_3$}}{2}
  		   {gl}{{g$_l$}}{2}
  		   {gr}{{g$_r$}}{2}
  		   {DFSi }{{DFS$_{i}$}}{3}
  		   {DFSbi}{{DFS$_{bi}$}}{3}
  		   {DFSf }{{DFS$_{f}$}}{3}
}
%% scheme-list .

% metadata

\title{Towards a miniKanren with fair search strategies}
\author{Kuang-Chen Lu}
\affiliation{Indiana University}
\author{Weixi Ma}
\affiliation{Indiana University}
\author{Daniel P. Friedman}
\affiliation{Indiana University}

\newcommand{\conde}{\texttt{cond$^e$}}
\newcommand{\conj}{\texttt{conj}}
\newcommand{\disj}{\texttt{disj}}
\newcommand{\clisting}[1]{
\begin{center}
  \begin{tabular}{c}
	\lstinputlisting{#1}
    \end{tabular}
\end{center}
}

\newcommand{\disjtwo}{\texttt{disj$_2$}}
\newcommand{\conjtwo}{\texttt{conj$_2$}}
\newcommand{\veryrecursiveo}{\texttt{very-recursive$^o$}}
\newcommand{\appendo}{\texttt{append$^o$}}
\newcommand{\reverso}{\texttt{revers$^o$}}
\newcommand{\repeato}{\texttt{repeat$^o$}}
\newcommand{\runstar}{\texttt{run$^*$}}
\newcommand{\Space}{\texttt{Space}}
\newcommand{\appendInf}{\texttt{append$^\infty$}}
\newcommand{\appendInfFair}{\texttt{append$^\infty_{fair}$}}
\newcommand{\appendMapInf}{\texttt{append-map$^\infty$}}
\newcommand{\appendMapInfFair}{\texttt{append-map$^\infty_{fair}$}}
\newcommand{\sInf}{\texttt{s$^\infty$}}
\newcommand{\tInf}{\texttt{t$^\infty$}}

\newcommand{\DFSi }[0]{DFS$_{i}$}
\newcommand{\DFSf }[0]{DFS$_{f}$}
\newcommand{\DFSbi}[0]{DFS$_{bi}$}
\newcommand{\BFS}[0]{BFS}
\newcommand{\BFSser}[0]{the current BFS}
\newcommand{\BFSse}[0]{current BFS}
\newcommand{\BFSimp}[0]{the improved BFS}
\newcommand{\BFSim}[0]{improved BFS}



\newtheorem{defn}{Definition}[section]

\begin{document}

\NOTE[LKC]{submission deadline: Mon 27 May 2019}

\begin{abstract}

We describe fairness levels in disjunction and conjunction
implementations.  Specifically, a disjunction implementation can be
fair, almost-fair, or unfair. And a conjunction implementation can be
fair or unfair.  We compare the fairness level of four search
strategies: the standard miniKanren interleaving depth-first search,
the balanced interleaving depth-first search, the fair depth-first
search, and the standard breadth-first search.  The two non-standard
depth-first searches are new. And we present a new, more efficient
and shorter implementation of the standard breadth-first search.
Using quantitative evaluation, we argue that each depth-first search is a
competitive alternative to the standard one, and that our improved 
breadth-first search implementation is more efficient than the
current one.

% We describe fairness levels in disjunction and conjunction
% implementations.  Specifically, a disjunction implementation can be
% fair, almost-fair, or unfair. And a conjunction implementation can
% be fair or unfair.  We compare the fairness level of four search
% strategies: the standard miniKanren interleaving depth-first search (\DFSi),
% the balanced interleaving depth-first search (\DFSbi), the fair depth-first
% search (\DFSf), and the standard breadth-first search (\BFSser).
% \DFSbi{} and \DFSf{} are new. And we present a new, more efficient
% and simpler implementation of \BFSser. Using quantitative evaluation, 
% we argue that
% \DFSbi{} and \DFSf{} are competitive alternatives to \DFSi, 
% and that our \BFSser{} implementation is more efficient than the current one.

% The syntax of a programming language should reflect its semantics. When 
%writing 
% a \conde{} expression in miniKanren, a programmer would expect all clauses 
% share the same chance of being explored, as these clauses are written in 
% parallel. The existing search strategy, interleaving depth-first search 
% (\DFSi{}), 
% however, prioritizes its clauses by the order how they are written down. 
% Similarly, when a \conde{} is followed by another goal conjunctively, a 
% programmer would expect states in parallel share the same chance of being 
% explored. Again, the answers by \DFSi{} is different from the 
% expectation. We have devised three new search strategies that have different 
% level of fairness in \disj{} and \conj{}.
% 

\end{abstract}

\maketitle

\section{introduction}

miniKanren is a family of relational programming languages.
\citet{Friedman:2005:RS:1121583,friedman_reasoned_2018} introduce
miniKanren and its implementation in \emph{The Reasoned Schemer} 
and \emph{The Reasoned Schemer, 2nd Ed} (TRS2). Although originally written in 
Scheme, miniKanren has been ported to many other languages (e.g. 
OCanren\citep{kosarev2018typed}). \citet{byrd2017unified} have 
demonstrated that miniKanren programs are useful in solving several difficult
problems. miniKanren.org contains the seeds of many difficult problems and 
their solutions. 

\NOTE[DPF]{``current'' was for BFSser if we need it, which we probably won't
since the re-emergence of BFSser.  So let's reserve the word just in case.}
\NOTE[LKC]{What do you mean by ``re-emergence''?}

A subtlety arises 
when a \conde{} contains many clauses: not every clause has an 
equal chance of contributing to the result. As an example, consider the 
following 
relation \repeato{} and its invocation. 

\clisting{Figures/repeato.rkt}

\noindent Next, consider the following disjunction of invoking \repeato{} with 
four 
different letters.

\clisting{Figures/example.rkt}

\noindent \conde{} intuitively relates its clauses with logical \texttt{or}. And 
thus an 
unsuspicious beginner would expect each letter to contribute equally to the 
result, as follows.

\clisting{Figures/run-repeato-fair.rkt}

\noindent The \conde{} in TRS2, however, generates a less expected result.

\clisting{Figures/run-repeato-idfs.rkt}

The miniKanren in TRS2 implements \textbf{i}nterleaving DFS (\DFSi), the cause of this 
unexpected result. With this search strategy, each \conde{} clause takes half 
of its received computational resources and passes the other half to its 
following clauses, except for the last clause that takes all resources it 
receives. In the example above, the \texttt{a} clause takes half of all 
resourses. And the \texttt{b} clause takes a quarter. Thus \texttt{c} and 
\texttt{d} barely contribute to the result.

%In the example above, the letters \texttt{c} and \texttt{d} are 
%allocated with less resources than \texttt{a}, and thus \texttt{c} and 
%\texttt{d} barely contribute to the result.

\DFSi{} is sometimes powerful for an expert. By carefully organizing the order 
of \conde{} clauses, a miniKanren program can explore more ``interesting'' 
clauses than those uninteresting ones, and thus use computational resources 
efficiently.

% A little miniKanrener, however, may beg to differ--understanding
% implementation details and fiddling with clause order is not the first
% priority of a beginner.

\DFSi{} is not always the best choice. For instance, it might be less 
desirable for novice miniKanren users---understanding implementation details 
and fiddling with clause order is not their first priority. 
There is another reason that miniKanren could use more search strategies than
just \DFSi. In many applications, there does not exist one order that serves all
purposes. For example, a relational dependent type checker contains
clauses for constructors that build data and clauses for eliminators that use
data. When the type checker is generating simple and shallow programs,
the clauses for constructors had better be at the top of the
\conde{} expression.
When performing proof searches for complicated programs, the clauses for 
eliminators had better be at the top of the \conde{} expression. With \DFSi, 
these two uses cannot be efficient at the same time. In fact, to make one use 
efficient, the other one must be more sluggish.

The specification that gives every clause in the same \conde{} equal 
``search priority'' is fair \disj{}. And search strategies with 
almost-fair \disj{} give every clause similar priority. 
Fair \conj{}, a related concept, is more subtle. We cover it in the next 
section.

To summarize our contributions, we
\begin{itemize}
	\item propose and implement \textbf{b}alanced \textbf{i}nterleaving 
depth-first search (\DFSbi{}), a new search strategy with almost-fair \disj{}.
	\item propose and implement \textbf{f}air depth-first search (\DFSf{}), 
a new search strategy with fair \disj{}.
	\item implement in a new way the standard breath-first search (\BFS) by 
\citet{seres1999algebra}, a search strategy with fair \disj{} and fair \conj{}. 
We refer to ``\BFSser{}'' as their implementation and ``the improved 
BFS'' as our new one. We formally prove that the two implementations are 
semantically equivalent, however, \BFSimp{} runs faster in all benchmarks and 
is shorter.
\end{itemize}

\section{search strategies and fairness}

In this section, we define fair \disj{}, almost-fair \disj{} and fair \conj{}. 
Before going further into fairness, we give a short review of the terms:
\emph{state}, \emph{space}, and \emph{goal}.
A \emph{state} is a collection of constraints. (Here, we restrict 
constraints to unification constraints.) Every answer corresponds to a 
state. A \emph{space} is a collection of states. And a \emph{goal} is a 
function from a state to a space.

Now we elaborate fairness by running more queries about \repeato{}. We never 
use \runstar{} in this paper because fairness is more interesting when we 
have an unbounded number of answers. It is perfectly fine, however, to use 
\runstar{} with any search strategy.

\subsection{fair \texttt{disj}}

Given the following program, it is natural to expect lists of each
letter to constitute $1/4$ in the answer list. \DFSi, TRS2's search
strategy, however, results in many more lists of \texttt{a}s than
lists of other letters. And some letters (e.g. \texttt{c} and
\texttt{d}) are rarely seen. The more clauses, the worse the situation.

\begin{center}
	\begin{tabular}{c}
		\lstinputlisting{Figures/repeato-disj-DFSi.rkt}
	\end{tabular}
\end{center}

Under the hood, the \conde{} here allocates computational resources to 
four trivially different spaces. The unfair \disj{} in 
\DFSi{} allocates many more resources to the first space. On the 
contrary, fair \disj{} would allocate resources evenly to each space. 

\begin{center}
	\begin{tabular}{l|r}
		\lstinputlisting{Figures/repeato-disj-DFSf.rkt} &
		\lstinputlisting{Figures/repeato-disj-BFS.rkt}
	\end{tabular}
\end{center}

Running the same program again with almost-fair \disj {} (e.g. 
\DFSbi{}) gives the same result. Almost-fair, however, is not 
completely fair, as shown by the following example. 

\begin{center}
	\begin{tabular}{c}
		\lstinputlisting{Figures/repeato-disj-DFSbi.rkt}
	\end{tabular}
\end{center}

\DFSbi{} is fair only when the number of goals is a power of 2, 
otherwise, it allocates some goals with twice as many resources as the 
others. In the above example, where the \conde{} has five clauses, \DFSbi{} 
allocates more resources to the clauses of \texttt{b}, \texttt{c}, and 
\texttt{d}.

We end this subsection with precise definitions of all levels of 
\disj{} fairness. Our definition of \emph{fair} \disj{} is slightly 
more general
than the one in \citet{seres1999algebra}, which is only
for binary disjunction. We generalize it to a multi-arity one.

\begin{defn}[fair \disj{}]
A \disj{} is fair if and only if it allocates computational resources evenly to 
spaces produced by goals in the same disjunction 
(i.e., clauses in the same \conde).
\end{defn}

\begin{defn}[almost-fair \disj{}]
A \disj{} is almost-fair if and only if it allocates computational resources
so evenly to spaces produced by goals in the same disjunction that 
the maximal ratio of resources is bounded by a constant.
\end{defn}

\begin{defn}[unfair \disj{}]
A \disj{} is unfair if and only if it is not almost-fair.
\end{defn}

\subsection{fair \texttt{conj}}
\label{sec:fairconj}

Given the following program, it is natural to expect lists of each letter to
constitute $1/4$ in the answer list. Search strategies with unfair \conj{}:
\DFSi, \DFSbi, and \DFSf, however, results in many more lists of \texttt{a}s 
than 
lists of other letters. And some letters are rarely seen. Here again, as the
number of clauses grows, the situation worsens. 

Although some strategies have a different level of fairness in \disj{}, they 
have the same behavior when there is no call to a relational definition in 
\conde{} clauses (including this case).

\begin{center}
\begin{tabular}{l|c|r}
    \lstinputlisting{Figures/repeato-conj-DFSi.rkt} &
    \lstinputlisting{Figures/repeato-conj-DFSf.rkt} &
    \lstinputlisting{Figures/repeato-conj-DFSbi.rkt} \\
\end{tabular}
\end{center}

Under the hood, the \conde{} and the call to \repeato{} are connected by 
\conj{}. The \conde{} goal outputs a space including four trivially 
different states. 
Applying the next conjunctive goal, \texttt{(repeato x q)}, produces four 
trivially different spaces.
In the examples above, all search strategies allocate more computational 
resources to the space of \texttt{a}. On the contrary, fair \conj{} 
would allocate resources evenly to each space. For example,

\begin{center}
	\begin{tabular}{c}
		\lstinputlisting{Figures/repeato-conj-BFS.rkt}
	\end{tabular}
\end{center}

A more interesting situation is when the first conjunct produces an unbounded
number of states. Consider the following example, a naive specification of 
fair \conj{} 
might require search strategies to produce all sorts of singleton lists, but 
there
would not be any lists of length two or longer, which makes the strategies 
incomplete. 
A search strategy is \emph{complete} if and only if ``every correct answer 
would be discovered after some finite time'' \cite{seres1999algebra}, 
otherwise, it is \emph{incomplete}. In the 
context of miniKanren, a search strategy is complete means that every correct 
answer has a position in large enough answer lists.

\begin{center}
	\begin{tabular}{c}
		\lstinputlisting{Figures/repeato-conj-infinite-naive.rkt}
	\end{tabular}
\end{center}

Our solution requires a search strategy with \emph{fair} \conj{} to organize
states in buckets in spaces, where each bucket contains finite states, 
and 
to allocate resources evenly among spaces derived from states in the 
same bucket. It is up to a search strategy designer to decide by what criteria 
to 
put states in the same bucket, and how to allocate resources among spaces 
related to different buckets.

\BFS{} puts states of the same cost in the same bucket, and allocates
resources carefully among spaces related to different buckets such
that it produces answers in increasing order of cost. The \emph{cost}
of an answer is its depth in the search tree (i.e., the number of
calls to relational definitions required to find the answer) 
\citep{seres1999algebra}. In the above examples, the cost of each answer is 
equal to their lengths because we need to apply \repeato{} $n$ times to find an 
answer of length $n$. In the following example, every answer is a list of a 
list of symbols, where inner lists in the same outer list are identical. Here 
the cost of each answer is equal to the length of its inner lists plus the 
length of its outer list. For example, the cost of \texttt{((a) (a))} is $1 + 2 
= 3$.

\begin{center}
	\begin{tabular}{c}
		\lstinputlisting{Figures/repeato-conj-infinite-sof.rkt}
	\end{tabular}
\end{center}

We end this subsection with precise definitions of all levels of \conj{} 
fairness.

\begin{defn}[fair \conj{}]
A \conj{} is fair if and only if it allocates computational resources evenly to 
spaces produced from states in the same bucket. A bucket is a finite 
collection of states. And search strategies with fair \conj{} should represent 
spaces with possibly unbounded collections of buckets. 
\end{defn}

\begin{defn}[unfair \conj{}]
A \conj{} is unfair if and only if it is not fair.
\end{defn}





\section{interleaving depth-first search}

\begin{figure}
	\lstinputlisting{Figures/DFSi-0.rkt}
	\caption{implementation of \DFSi{} (Part I)}
	\label{DFSi-0}
\end{figure}

\begin{figure}
	\lstinputlisting{Figures/DFSi-1.rkt}
	\caption{implementation of \DFSi{} (Part II)}
	\label{DFSi-1}
\end{figure}

In this section, we review the implementation of interleaving depth-first 
search (\DFSi). We focus on parts that are relevant to this paper. TRS2,
chapter 10 and the appendix, ``Connecting the wires'', 
provide a comprehensive description of the miniKanren 
implementation but limited to unification constraints ($\equiv$).
Fig.~\ref{DFSi-0} and Fig.~\ref{DFSi-1} show parts that are later compared 
with other search strategies. We follow some conventions to name variables: 
\texttt{s}s name states; \texttt{g}s (possibly with subscript) name goals; 
variables ending with $^\infty$ name (search) spaces. Fig.~\ref{DFSi-0} shows 
the 
implementation of \disj. The 
first function, \disjtwo, implements binary disjunction. It applies the 
two disjunctive goals to the input state \texttt{s} and composes the two 
resulting spaces with \appendInf{}. The following syntax 
definitions say \disj{} is right-associative. Fig.~\ref{DFSi-1} 
shows the implementation of \conj. The first function, \conjtwo{}, 
implements binary conjunction. 
It applies the \emph{first} goal to the input state, then applies the second 
goal to states in the resulting space. The helper function 
\appendMapInf{} applies its input goal to states 
in its input space and composes the resulting spaces. It reuses 
\appendInf{} for space composition. The following syntax 
definitions say \conj{} is also right-associative.

\section{balanced interleaving depth-first search}

\begin{figure}
	\lstinputlisting{Figures/DFSbi.rkt}
	\caption{implementation of \DFSbi{}}
	\label{balanced-disj}
\end{figure}

Balanced interleaving DFS (\DFSbi{}) has an almost-fair \disj{} and unfair 
\conj{}. The implementation of \DFSbi{} differs from 
\DFSi{}'s in the \disj{} macro. When there are one or more disjunctive 
goals, the new \disj{} builds a balanced binary tree whose leaves are the goals 
and 
whose nodes are \disjtwo{}s, hence the name of this search strategy. 
In contrast, the \disj{} in \DFSi{} constructs the binary tree in a 
particularly unbalanced form.
We list the new \disj{} with its helper in Fig.~\ref{balanced-disj}.
The new helper, \texttt{disj+}, takes two additional `arguments'. They 
accumulate goals to be put in the left and right subtrees. The first clause 
handles the case where there is only one goal. In this case, the tree is the 
goal itself. When there are more goals, we partition the list of goals 
into two sublists of roughly equal lengths and recur on the two sublists. We 
move goals to the accumulators in the last clause. As we are moving 
two goals each time, there are two base cases: (1) no goal remains; (2) one 
goal remains. We handle these two new base cases in the second clause and the 
third clause, respectively. 

\section{fair depth-first search}

\begin{figure}
	\lstinputlisting{Figures/DFSf.rkt}
	\caption{implementation of \DFSf{}}
	\label{fDFS}
\end{figure}

Fair DFS (\DFSf) has fair \disj{} and unfair \conj{}. The 
implementation of \DFSf{} differs from \DFSi{}'s in 
\disjtwo{} (Fig.~\ref{fDFS}). The new \disjtwo{} calls a new and 
fair version of \appendInf{}. \appendInfFair{} 
immediately 
calls 
its helper, \texttt{loop}, with the first argument, \texttt{s?}, set to 
\texttt{\#{}t}, which indicates that we haven't swapped
\sInf{} and \tInf{}. The swapping 
happens at 
the third \texttt{cond} clause in the helper, where \texttt{s?} is updated 
accordingly. The first two \texttt{cond} clauses essentially copy the 
\texttt{car}s and stop recursion when one of the input spaces is obviously 
finite. The third clause, as we mentioned above, is just for swapping. When the 
fourth and last clause runs, we know that both \sInf{} and 
\tInf{} are ended with a thunk, and that we have swapped them. In 
this case, we construct a new thunk. The new thunk swaps back the two spaces in 
the
recursive call to \texttt{loop}. This is unnecessary for fairness---we do it to 
produce answers in a more readable order.

\section{breadth-first search}

\begin{figure}
	\lstinputlisting{Figures/BFSuni.rkt}	
	\caption{stepping-stone toward \BFSimp{} (based on \DFSf{})}
	\label{BFSuni}
\end{figure}

\BFS{} has both fair \disj{} and fair \conj{}. Our first BFS implementation 
(Fig.~\ref{BFSuni})
serves as a ``stepping-stone'' toward \BFSimp{}. It is so 
similar to \DFSf{} 
(not \DFSi{}) that we only need to apply two changes: (1) rename 
\appendMapInf{} 
to \appendMapInfFair{} and (2) replace \appendInf{} with \appendInfFair{} in 
\appendMapInfFair{}'s body. 

This implementation can be improved in two ways. First, as mentioned in 
\autoref{sec:fairconj}, \BFS{} puts answers in buckets and answers of the same 
cost are in the same bucket. In the above implementation, however, 
it is not obvious how we manage cost information---the \texttt{car}s of a 
space have cost 0 (i.e., they are all in the same bucket), and every 
thunk indicates an increment in cost. It is even more subtle that 
\appendInfFair{} and the \appendMapInfFair{} respects the cost information. 
Second, \appendInfFair{} is extravagant in memory usage. It makes $O(n+m)$ new 
\texttt{cons} cells every time, where $n$ and $m$ are the sizes of 
the first buckets of two input spaces. \DFSf{} is also space extravagant.

In the following subsections, we first describe \BFSimp{} implementation that 
manages cost information in a more clear and concise way and is less 
extravagant in memory usage. Then we compare \BFSimp{} with \BFSser{}.

\subsection{improved BFS}

\begin{figure}
	\lstinputlisting{Figures/BFSimp.rkt}	
	\caption{New and changed functions in \BFSimp{} that implements pure 
	features}
	\label{BFSimp}
\end{figure}

We simplify the cost information by changing the \texttt{Space} type, 
modifying related function definitions, and introducing a few more functions.

The new type of \texttt{Space} is a pair whose \texttt{car} is a list of 
answers (the bucket), and whose \texttt{cdr} is either \texttt{\#{}f} or a 
thunk returning a space. A falsy \texttt{cdr} means the space is obviously 
finite. 

We list functions related to the pure subset in Fig.~\ref{BFSimp}. The first 
three functions are space constructors. \texttt{none} makes an empty space; 
\texttt{unit} makes a space from one answer; and \texttt{step} makes a space 
from a thunk. The remaining functions are as before. 
%We compare these functions 
%with \BFSser{} in our proof. 

Luckily, the change in \appendInfFair{} also fixes the miserable space 
extravagance---the use of \texttt{append} helps us to reuse the first 
bucket of \tInf{}.

\citet{kiselyov2005backtracking} have demonstrated that a \emph{MonadPlus} 
hides in implementations of logic programming systems. \BFSimp{} is not an 
exception: \appendMapInfFair{} is like \texttt{bind}, but takes 
arguments in reversed order; \texttt{none}, \texttt{unit}, and 
\appendInfFair{} correspond to \texttt{mzero}, \texttt{unit}, and 
\texttt{mplus}, respectively.

\begin{figure}
	\lstinputlisting{Figures/BFSimp-cont.rkt}	
	\caption{New and changed functions in \BFSimp{} that implement impure 
		features}
	\label{BFSimp-cont}
\end{figure}

Functions implementing impure features are in Fig.~\ref{BFSimp-cont}. The 
first function, \texttt{elim}, takes a space \sInf{} and two 
continuations \texttt{fk} and \texttt{sk}. When \sInf{} contains 
no answers, it calls \texttt{fk}. Otherwise, it calls 
\texttt{sk} with the first answer and the rest of the space. This function is 
similar to an eliminator of spaces, hence the name. The remaining 
functions are as before.

\subsection{compare \BFSimp{} with \BFSser{}}

In this subsection, we compare the pure subset of \BFSimp{} with \BFSser{}. We 
focus on the pure subset because \BFSser{} is designed for a pure 
relational programming system. We prove in Coq that these two search strategies 
are semantically equivalent, since the result of \texttt{(run n ? g)} is 
the same either way. (See supplements for the formal proof.) To 
compare efficiency, we translate \BFSser{}'s Haskell code into Racket (See 
supplements for the translated code). The translation is direct due to the 
similarity of the two relational programming systems. The translated code is 
longer than \BFSimp{}. And it runs slower in all benchmarks. Details about 
differences in efficiency are in \autoref{sec:quan}.

\section{quantitative evaluation}
\label{sec:quan}

\begin{table}
	% run in command line, with a memory limit of 500MB
	\begin{tabular}{|c|c|c|c|c|c|c|}
		\hline 
	benchmark & size & \DFSi & \DFSbi & \DFSf & \BFSim & \BFSse	\\
		\hline
		\veryrecursiveo & 100000 & 166 & 103 & 412 &   451 &   1024 \\
	    				& 200000 & 283 & 146 & 765 &   839 &   1875 \\
						& 300000 & 429 & 346 &2085 &  1809 &   4408 \\
		\hline  
		\appendo        & 	 100 &  17 &  18 &  17 &    17 &     65 \\
		         	  	& 	 200 & 145 & 137 & 137 &   142 &    121 \\
          		      	&	 300 & 388 & 384 & 387 &   429 &    371 \\
%   		& - & - & - & - & - & - \\
		\hline 
		\reverso		& 	  10 &   3 &   4 &   3 &    18 &     36 \\
						& 	  20 &  35 &  35 &  33 &  3070 &   3695 \\
						& 	  30 & 329 & 333 & 315 & 75079 & 107531 \\
		\hline
		quine-1 		& 1 & 13 & 14 & 10 & - & - \\
						& 2 & 30 & 34 & 25 & - & - \\
						& 3 & 41 & 48 & 33 & - & - \\
		\hline
		quine-2 		& 1 & 22 & 21 & 12 & - & - \\
						& 2 & 51 & 46 & 24 & - & - \\
						& 3 & 72 & 76 & 32 & - & - \\
		\hline
		'(I love you)-1 &  99 &  10 &  37 & - &   40 &   96 \\
						& 198 &  22 &  71 & - &  374 &  652 \\
						& 297 &  24 & 292 & - & 1668 & 3759 \\
		\hline
		'(I love you)-2 &  99 & 485 & 179 & - &   44 &   94 \\
						& 198 & 808 & 638 & - &  414 &  630 \\
						& 297 &1265 & 916 & - & 1651 & 3691 \\ 
		\hline 
		% The data below is test with old benchmarks
%		quine-1 & 1 &  71 &  44 & 69 & ??? & - & - \\ 
%		& 2 & 127 & 142 & 95 & ??? & - & - \\ 
%		& 3 & 114 & 114 & 93 & ??? & - & - \\ 
%		\hline
%		quine-2 & 1 & 147 & 112 &  56 & ??? & - & - \\ 
%		& 2 & 161 & 123 & 101 & ??? & - & - \\ 
%		& 3 & 289 & 189 & 104 & ??? & - & - \\ 
%		\hline 
%'(I love you)-1 &  99 & 56 & 15 & 22 & ??? &  74 & 165 \\ 
%& 198 & 53 & 72 & 55 & ??? &  47 &  74 \\
%& 297 & 72 & 90 & 44 & ??? & 181 & 365 \\ 
%\hline
%'(I love you)-2 &  99 & 242 &  61 & 16 & ??? &  66 &  99 \\ 
%& 198 & 445 & 110 & 60 & ??? &  42 &  64 \\
%& 297 & 476 & 146 & 49 & ??? & 186 & 322 \\ 
%\hline 
	\end{tabular}
	\caption{The results of a quantitative evaluation: running times of 
	benchmarks in milliseconds}
	\label{compare-efficiency}
\end{table}

In this section, we compare the efficiency of search strategies. A concise 
description is in Table~\ref{compare-efficiency}. A hyphen means ``running out
of 500 MB memory''. The first two benchmarks are from 
TRS2. \reverso{} is from 
\citet{rozplokhas2018improving}. 
The next two benchmarks about quine are based on a similar test case in 
\citet{byrd2017unified}. We modify the relational interpreters because we don't 
have disequality constraints (e.g. \texttt{absent$^o$}). The sibling 
benchmarks differ in the \conde{} clause order of their relational 
interpreters.
The last two benchmarks are about synthesizing expressions that 
evaluate to \texttt{'(I love you)}. 
They are also based on a similar test case in \mbox{\citet{byrd2017unified}}. 
Again, 
we modify the relational interpreters for the same reason. And the sibling 
benchmarks differ in the \conde{} clause order of their relational 
interpreters. The first one has elimination rules (i.e. application, 
\texttt{car}, and \texttt{cdr}) at the end, while the other has them at the 
beginning. We conjecture that \DFSi{} would perform badly in the second case 
because elimination rules complicate the problem when synthesizing (i.e., our 
evaluation supports our conjecture.)

In general, only \DFSi{} and \DFSbi{} constantly perform well. \DFSf{}
is just as efficient in \appendo, \reverso, and \texttt{quine}s. All BFS
implementations have obvious overhead in most cases. 
Among the three variants of DFS, which all have unfair \conj{}, \DFSf{} is most 
resistant to clause permutation in \texttt{quines}, followed by \DFSbi{} then 
\DFSi{}. \DFSf{}, however, runs out of memory in \texttt{'(I love you)}s. In 
contrast, \DFSbi{} is still more stable than \DFSi{}. Thus, we consider 
\DFSbi{} a competitive alternative to \DFSi{}.
Among the two implementations of \BFS, \BFSimp{} constantly performs as well 
or better. 
It is hard to conclude how \disj{} fairness affects performance based on these 
statistics.
Fair \conj{}, however, imposes considerable overhead constantly except in 
\appendo. The reason might be that those strategies tend to keep more 
intermediate answers in the memory.

\section{related works}

\citet{yang2010adventures} points out that a disjunct complex would be `fair' 
if it were a full and balanced tree.

\citet{seres1999algebra} describe a breadth-first search 
strategy. We present another implementation. Our implementation is semantically 
equivalent to theirs. But, ours is shorter and performs better in comparison 
with a straightforward translation of their Haskell code.

\citet{rozplokhas2018improving} address the non-commutativity of conjunction, 
while our work about \disj{} fairness addresses the non-commutativity of 
disjunction.

\section{conclusion}

We analyze the definitions of fair \disj{} and fair \conj{}, then propose a 
new definition of fair \conj{}. Our definition of fair \conj{} is orthogonal 
with completeness.

We devise two new search strategies (i.e., balanced interleaving DFS 
(\DFSbi{}) and fair DFS (\DFSf{})) and devise a new 
implementation of \BFS. These strategies have different features 
in fairness: \DFSbi{} has an almost-fair \disj{} and unfair \conj{}. 
\DFSf{} has fair \disj{} and unfair \conj{}. \BFS{} has both fair
\disj{} and fair \conj{}.

Our quantitative evaluation shows that \DFSbi{} is a competitive 
alternative to \DFSi{}, the current miniKanren search strategy,
and that \BFSimp{} is more practical than \BFSser{}. 

% \BFSimp{}
We prove formally that \BFSimp{} is semantically equivalent to \BFSser{}. But, 
\BFSimp{} is shorter and performs better in comparison with a straightforward 
translation of their Haskell code.

\NOTE[DPF]{Please leave ``prove'' (just here).}
\NOTE[LKC]{OK. I won't touch it.}

We acknowledge that we are making large claims about efficiency given that we 
have just a few benchmarks, but our intuition is based on sound principles. ...

\NOTE[DPF]{Finish this paragraph with a clarification and you may
place this anywhere in the conclusion. This may require some consultation with 
Weixi.}
\NOTE[LKC]{I have sent an email to Weixi on the clarification.}

\section*{acknowledgments}

\bibliographystyle{ACM-Reference-Format}
\bibliography{citation}

\end{document}

